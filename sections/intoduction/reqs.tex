\section{Requerimientos}
\label{intro:sec:reqs}

En el mundo del desarrollo del software es bastante habitual dividir los requisitos en dos categorías: los requisitos funcionales y los requisitos no funcionales. Los primeros describen las funcionalidades que debe tener el sistema, centrándose en acciones, comportamientos e interacciones específicas del usuario. Por otro lado, los segundos describen las características no funcionales del sistema, como el rendimiento, la escalabilidad o la seguridad.

Empezando por los requisitos funcionales, se identifican los siguientes:

\begin{itemize}
    \item \textbf{RF1 - Gestión de pedidos:} Los usuarios deben poder gestionar los pedidos de los clientes, lo que incluye la creación, edición y eliminación de pedidos. Además, deben poder asignar estos pedidos a diferentes canales de venta en línea y editar sus atributos, como estado del pedido, fecha de entrega y detalles del cliente.
    \item \textbf{RF2 - Gestión de productos:} Los usuarios deben poder añadir, editar y eliminar productos de la plataforma. Adicionalmente deben poder asignar estos productos a diferentes canales de venta en línea y editar sus atributos, como precio, stock e imágenes.
    \item \textbf{RF3 - Gestión de envíos:} Los usuarios deben poder gestionar los envíos de los pedidos, lo que incluye la asignación de métodos de envío, seguimiento de envíos y actualización del estado de los envíos.
    \item \textbf{RF4 - Integración de herramientas básicas de análisis:} Los usuarios deben poder acceder a herramientas básicas de análisis que les permitan visualizar el rendimiento de sus ventas, pedidos y productos. Esto incluye gráficos y estadísticas sobre ventas, pedidos y productos.
\end{itemize}

Respecto a los requisitos no funcionales, se identifican los siguientes:

\begin{itemize}
    \item \textbf{RNF1 - \textit{Stack} tecnológico:} La plataforma debe ser desarrollada utilizando tecnologías modernas y populares, como React para el \textit{frontend}, Django para el \textit{backend} y PostgreSQL para la base de datos.
    \item \textbf{RNF2 - Escalabilidad:} El diseño de la plataforma debe permitir añadir nuevas funcionalidades y canales de venta en el futuro sin necesidad de cambios significativos en la estructura de la base de datos o de la lógica de ninguna de las partes de la aplicación.
    \item \textbf{RNF4 - Usabilidad:} La plataforma debe ser intuitiva y fácil de usar, permitiendo a los usuarios gestionar sus productos, pedidos y envíos de manera sencilla. Se pone un especial énfasis en la automatización de procesos para los comerciantes, eliminando la necesidad de personal técnico especializado.
    \item \textbf{RNF5 - Rendimiento:}  La plataforma debe operar con un buen rendimiento para que los usuarios puedan trabajar de manera fluida y sin problemas. Esto incluye la optimización de la entrega de datos entre el \textit{frontend} y el \textit{backend}, así como la gestión eficiente de los recursos del servidor.
    \item \textbf{RNF6 - Mantenibilidad:} El código y la arquitectura de la plataforma deben estar diseñados para facilitar el mantenimiento, permitiendo corregir errores y añadir nuevas funcionalidades de manera sencilla.
\end{itemize}