\section{Estado del Arte}
\label{intro:sec:estado_arte}

El comercio electrónico global ha experimentado un crecimiento extraordinario, alcanzando los 6.15 billones de dólares en 2024, lo que representa más del 21.5\% de las ventas minoristas totales a nivel mundial. Este sector continuará expandiéndose hasta los 7.39 billones de dólares para finales de 2025, representando el 23.6\% de todas las ventas minoristas \cite{ea_bluehost}. Por lo que hace España, el comercio electrónico superó los 23114 millones de euros en el segundo trimestre de 2024, registrando un crecimiento interanual del 12,8\% \cite{ea_cnmc}.

En este contexto, los consumidores realizan en promedio el 42\% de sus compras online a través de marketplaces, y el 92\% de los consumidores estadounidenses esperan usar marketplaces al mismo nivel o más en el futuro \cite{ea_future}. Sin embargo, la complejidad operacional representa el principal obstáculo para las empresas que operan en múltiples marketplaces. Los retailers enfrentan desafíos significativos relacionados con sistemas fragmentados, falta de integración de APIs propietarias y ausencia de visibilidad de inventario en tiempo real. La gestión manual de pedidos se vuelve prácticamente imposible cuando aumentan los volúmenes o cuando se opera en varias plataformas simultáneamente. Sin una visión general centralizada, aumenta la posibilidad de errores en el mantenimiento del stock, incluyendo sobreventas y retrasos en actualización de pedidos \cite{ea_volo}.

Las aplicaciones omnicanal son una solución ya existente. De hecho, el mercado de soluciones omnicanales, que facilita la integración de múltiples canales de venta, estaba valorado en 5.96 mil millones de dólares en 2021 y, para finales de 2025, este mercado alcanzará los 10.13 mil millones de dólares, con proyecciones de llegar a 25.35 mil millones para 2032, exhibiendo una tasa de crecimiento anual del 14\% \cite{ea_grandview}\cite{ea_coherent}.

Sin embargo, muchas empresas encuentran grandes dificultades para entender e implementar correctamente este tipo de soluciones omnicanales. Aunque el 90\% de los expertos en marketing considera imprescindible tener una visión unificada del cliente multicanal, solo un 20\% de los comercios logra aplicarla realmente en su operativa diaria, lo que evidencia una brecha importante entre la teoría y la práctica \cite{ea_adyen}. Por este motivo, se presenta la necesidad de una plataforma que facilite el transcurso de integración y acerque mucho más al perfil no técnico a la implementación de una solución omnicanal.