\section{Justificación}
\label{intro:sec:justificacion}

Actualmente la venta en línea es una de las formas más comunes de comercio. La facilidad y la comodidad de poder comprar desde casa han hecho que cada vez más personas opten por esta forma de compra, pues se ofrece un catálogo mucho más amplio que en el comercio convencional al alcance de unos solos clics. Adicionalmente, la pandemia de la COVID-19 ha acelerado este proceso; muchas tiendas físicas se vieron obligadas a cerrar y optaron por empezar a vender sus productos en línea, donde no se requiere ni de un local físico ni de tanto personal para hacer llegar sus productos al cliente final.

Sin embargo, la venta en línea no es tan sencilla como puede parecer en un inicio. Existe una gran variedad de canales de venta en línea, como pueden ser los \textit{marketplaces} o las tiendas en línea, comúnmente llamadas \textit{e-commerce}. Cada uno de estos canales tiene sus propias características y un funcionamiento propio, lo que hace que cada uno de ellos tenga sus facilidades y complejidades.

Los gigantes Amazon, eBay, AliExpress, entre otros, se presentan como \textit{marketplaces} y son en gran parte responsables de que la competencia en la venta en línea sea mucho tan dura. Los llamados \textit{marketplaces} ofrecen una plataforma donde cualquier persona puede vender sus productos a cambio de una comisión por cada venta realizada. Esto ha hecho que muchos pequeños comercios se vean forzados a vender sus productos en estos \textit{marketplaces}, pues es la única forma de llegar a un público más amplio.

Por otro lado, soluciones como Shopify y WooCommerce ofrecen una plataforma para crear una tienda en línea de manera relativamente sencilla. Estas soluciones permiten a los comerciantes tener su propia tienda en línea, donde pueden vender sus productos sin necesidad de pasar por la comisión que toman los \textit{marketplaces}. No obstante, llegar a un público amplio es mucho más complicado, pues no se cuenta con la visibilidad que ofrece este otro canal.

Tanto en las tiendas en línea propias como en los \textit{marketplaces} existe una dificultad común: la elevada barrera de entrada para quienes no están familiarizados o interesados en el entorno digital. Para vender en línea, es necesario entender tanto el funcionamiento general de estas plataformas como ciertos aspectos técnicos, ya sea para montar una tienda propia o para integrar productos en un \textit{marketplace}.

Además, gestionar múltiples canales de venta suele ser una tarea compleja, ya que cada plataforma tiene sus propias reglas y formas de operar. Esto implica administrar productos, pedidos, clientes y envíos de forma separada, lo que ralentiza el proceso de venta y lo hace inviable cuando los volúmenes de venta aumenta.

Con todo esto, han surgido soluciones que permiten a los comercios gestionar sus productos en distintos canales de venta en línea, pero estas soluciones suelen ser complejas y requieren de un conocimiento técnico avanzado para su implementación. Por lo tanto, se hace necesario un enfoque más accesible y sencillo para los pequeños comercios que desean expandir su presencia en línea sin complicaciones técnicas.

Por ello, se plantea la creación de una plataforma que permita a los comercios centralizar y gestionar, de manera sencilla y eficiente, sus productos en distintos canales de venta en línea. Esta plataforma permitiría a los comercios tener una visión global de sus productos, de sus pedidos y de sus clientes en un único lugar. De este modo, un negocio podría expandir su mercado a distintos canales de venta en línea en pocos pasos y sin necesidad de tener conocimientos técnicos.
