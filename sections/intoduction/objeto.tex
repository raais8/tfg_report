\section{Objeto}
\label{intro:sec:objeto}

En los últimos años, el comercio electrónico se ha consolidado como una parte fundamental de la economía global. La capacidad de comprar y vender productos y servicios a través de internet ha transformado la interacción de las empresas con sus clientes, propiciando la aparición de nuevos canales de venta en línea que permiten llegar a un público más amplio y diversificado, a diferencia de la dependencia previa en tiendas físicas o distribuidores locales.

No obstante, el comercio electrónico no es tan sencillo como podría parecer en primera instancia y las empresas deben enfrentar una serie de desafíos para establecer y mantener una presencia efectiva en línea. Entre estos desafíos se encuentran la gestión de múltiples plataformas de venta, la sincronización de inventarios y de precios, la logística de envíos, entre muchos otros.

Para ayudar a las empresas a superar estos obstáculos, han surgido diversas soluciones, destacando las plataformas e-commerce y los marketplaces. Estas plataformas permiten a las empresas acceder a un mercado más amplio, como es el comercio en línea, de manera más sencilla y menos costosa. Sin embargo, la gestión de múltiples plataformas y marketplaces puede resultar complicada y consumir mucho tiempo, especialmente para las pequeñas y medianas empresas que no cuentan con los recursos necesarios para gestionar cada canal de venta de manera individual.

En este contexto, el objetivo principal de este proyecto es crear y desarrollar una plataforma web diseñada específicamente para permitir a los comerciantes centralizar la gestión de pedidos, productos y envíos en múltiples canales de venta en línea. Esta solución integral busca simplificar y optimizar la expansión de negocios al permitir a los comerciantes cargar sus productos en la plataforma central y, desde allí, publicarlos eficientemente en distintos canales de venta en línea. Además, la plataforma proporcionará herramientas para gestionar los pedidos y envíos de manera centralizada, lo que permitirá a los comerciantes ahorrar tiempo y recursos al evitar la necesidad de gestionar cada canal de venta por separado.