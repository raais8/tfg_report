\chapter{Análisis y valoración de las Implicaciones Ambientales y Sociales}
\label{ch:ambiental_social}

En esta sección se analizan y valoran los posibles impactos ambientales y sociales derivados del proyecto. Se busca analizar los efectos en las cuatro fases del ciclo de vida del proyecto: desarrollo, ejecución, vida útil y fin de vida. Para cada fase, se identifican los impactos ambientales y sociales, se proponen medidas de mitigación y se valoran.

Este análisis busca no solo detectar riesgos potenciales, sino también destacar oportunidades para contribuir positivamente a un desarrollo más sostenible y responsable en el ámbito del desarrollo de software y, en cierta medida, de la gestión de \textit{marketplaces}.


\section{Implicaciones Ambientales}
\label{as:sec:ambiental}

La evaluación de las implicaciones ambientales derivadas del proyecto debe identificar y valorar los impactos, tanto positivos como negativos, en el medio ambiente a lo largo de su ciclo de vida. Adicionalmente, debe estudiar las medidas a considerar para prevenir dichos impactos negativos y potenciar los positivos.

En las siguientes secciones se analizan y valoran las implicaciones ambientales en cada una de las fases del ciclo de vida del proyecto, así como las medidas de mitigación propuestas.

\subsection{Fase de desarrollo}

Al tratarse de un proyecto de desarrollo de software, el impacto ambiental más relevante en esta fase es el consumo de energía eléctrica por el uso de equipos informáticos. Sin embargo, de manera indirecta, este consumo lleva asociado un impacto ambiental derivado de la generación de energía eléctrica: la generación de gases de efecto invernadero.

Para mitigar el posible impacto derivado del consumo de energía eléctrica, se ha tratado de hacer un uso eficiente de los equipos informáticos, evitando el uso de equipos innecesarios, como pueden ser servidores de desarrollo, y apagando los equipos cuando no se están utilizando. Por otro lado, cabe remarcar que toda la documentación del proyecto se ha realizado de manera digital, evitando el uso de papel y reduciendo así el impacto ambiental asociado a la impresión de documentos.

Con todo esto presente, se puede estimar que los impactos ambientales en esta fase son controlables y relativamente bajos, únicamente asociados al consumo energético individual de mi persona.

\subsection{Fase de ejecución}

En esta segunda fase se contempla el impacto ambiental derivado a la implementación del software desarrollado a los medios de comunicación y almacenamiento necesarios para su funcionamiento. En este caso, al tratarse de una aplicación web, el impacto ambiental más relevante es el consumo energético asociado a los servidores que alojan la aplicación y a la infraestructura de red necesaria para su funcionamiento.

Para tratar de reducir al máximo este impacto, se contempla hacer una optimización del código, evitando el uso de recursos innecesarios y haciendo un uso eficiente de las consultas a bases de datos. Como mayor sea la eficiencia del código, menor será el consumo energético asociado a su ejecución.

Consecuentemente, se puede estimar que el impacto ambiental en esta fase es moderado, ya que depende de la eficiencia del código y de la infraestructura utilizada para alojar la aplicación.

\subsection{Vida útil}

En esta tercera fase, el paradigma ya cambia considerablemente. Al tratarse de una aplicación web, todo el software desarrollado debe ser ejecutado en un servidor, el cual debe estar encendido y funcionando las 24 horas del día. Esto implica un consumo energético constante, que puede ser significativo dependiendo de la infraestructura utilizada.

Para tratar de mitigar estos impactos, se contempla realizar una buena elección de proveedores de servicios en la nube, los cuales demuestren un compromiso con la sostenibilidad, haciendo uso de energías renovables y poseyendo certificaciones de eficiencia energética.

De esta manera, en la fase de vida útil hay cierto traslado de responsabilidad hacia el proveedor de servicios en la nube, ya que la sostenibilidad del proyecto dependerá en gran medida de las prácticas ambientales de dicho proveedor. Sin embargo, se debe tratar de hacer la elección más responsable posible, pues es la fase con un mayor impacto.

\subsection{Fin de vida}

En la fase final del ciclo de vida del proyecto, se debe considerar el impacto ambiental derivado de la eliminación del software y de la infraestructura utilizada para su funcionamiento. En este caso, el impacto más relevante es el asociado a los residuos electrónicos generados por los servidores y otros equipos utilizados para alojar la aplicación.

La mitigación de este impacto va también bastante ligada a la elección de proveedores de servicios en la nube, ya que muchos de ellos ofrecen servicios de reciclaje y reutilización de equipos electrónicos. Además, se debe considerar la posibilidad de migrar el software a nuevas infraestructuras más sostenibles, evitando así la generación de residuos innecesarios.

Así pues, el impacto ambiental en esta fase puede ser significativo, pero también es controlable y mitigable mediante la elección de proveedores responsables y prácticas de reciclaje adecuadas. Mantenerse actualizado con nuevos proveedores y tecnologías sostenibles puede ayudar a reducir el impacto ambiental en esta fase.

\section{Implicaciones Sociales}
\label{as:sec:sociales}

La evaluación de las implicaciones sociales del proyecto debe identificar y valorar los impactos, tanto positivos como negativos, en la sociedad a lo largo de su ciclo de vida. Debe valorar todos aquellos aspectos éticos y de género que puedan surgir durante el desarrollo del proyecto, así como los posibles impactos en la comunidad y en los usuarios finales de la aplicación. Por último, debe estudiar las medidas a considerar para prevenir dichos impactos negativos y potenciar los positivos.

En las siguientes secciones se analizan y valoran las implicaciones sociales en cada una de las fases del ciclo de vida del proyecto, así como las medidas de mitigación propuestas. Al no haber gran diferencia entre la fase de ejecución y la fase de vida útil, se tratarán conjuntamente.

\subsection{Fase de desarrollo}

En la fase de desarrollo, el impacto social es más bien limitado, ya que el proyecto se desarrolla de manera individual. Sin embargo, es importante considerar aspectos éticos y de género en el desarrollo del software, como la accesibilidad y la inclusión de diferentes grupos sociales.

Para mitigar posibles impactos negativos, se ha tratado de seguir buenas prácticas de desarrollo, como el uso de herramientas de control de versiones y la documentación del código. Además, se ha procurado que el software sea accesible y usable para diferentes grupos sociales, teniendo en cuenta aspectos como la diversidad funcional y la inclusión de personas con discapacidad.

\subsection{Fase de ejecución y Vida útil}

En esta fase, el impacto social puede ser más relevante. Se identifican diversos impactos sociales, principalmente positivos. La plataforma facilita la expansión de negocios al reducir la complejidad operativa de gestionar múltiples canales de venta, lo que favorece la inclusión de pequeños comerciantes sin conocimientos técnicos avanzados. Además, optimiza la gestión del trabajo, permitiendo que los empleados se centren en tareas de mayor valor estratégico. A nivel social, también contribuye a la mejora de la calidad de vida de los usuarios al simplificar procesos rutinarios, y refuerza la accesibilidad mediante una interfaz centrada en la usabilidad para todos los perfiles.

Para maximizar estos beneficios y mitigar los posibles riesgos, se pretende aplicar un conjunto de medidas para cada uno de los impactos. En materia de seguridad y ética, se pretende desarrollar un sistema de autenticación robusto que protege los datos sensibles de clientes y pedidos, reduciendo el riesgo de accesos no autorizados. Asimismo, se planea poner especial atención en evitar sesgos en la representación de datos y en asegurar la escalabilidad del sistema para adaptarse a nuevas funcionalidades sin excluir a nuevos usuarios.

En definitiva, el impacto de la plataforma durante su ejecución y operación puede considerarse positivo, tanto desde una perspectiva social como funcional. Su diseño centrado en la accesibilidad, la automatización y la seguridad promueve un entorno más equitativo y eficiente.

\subsection{Fin de vida}

En la fase final del ciclo de vida del proyecto, el principal impacto identificado está relacionado con el cese de operaciones de la aplicación y sus consecuencias para los usuarios. La interrupción del servicio podría generar dificultades para los comercios que dependan de la plataforma para gestionar sus operaciones diarias, además del riesgo potencial de pérdida de datos relevantes como historiales de productos, pedidos o información de clientes.

Para mitigar este impacto, se contempla la implementación de procedimientos claros de finalización del servicio. Estos incluirán la notificación con antelación suficiente a todos los usuarios para que puedan realizar una migración. En caso de que los usuarios no requieran esta migración, se garantizará la eliminación definitiva de la información, cumpliendo estrictamente con las normativas de protección de datos y privacidad vigentes.

En conjunto, aunque el fin de vida de la plataforma podría tener un impacto sensible en la operativa de los usuarios, una planificación anticipada y responsable permitiría gestionarlo de forma ética y controlada.
