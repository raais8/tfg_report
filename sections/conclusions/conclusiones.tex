\chapter{Conclusiones}
\label{chap:conclusiones}

La realización de este proyecto ha llevado al desarrollo de una plataforma web para la gestión centralizada de múltiples canales de venta en línea o \textit{marketplaces}. Consecuentemente, se ha permitido demostrar la viabilidad técnica y funcional de una herramienta que tiene capacidad de automatizar y simplificar una parte muy importante de las tareas de gestión de un negocio que opera en múltiples plataformas de venta. La solución desarrollada aborda problemáticas comunes tales como la fragmentación de la información, la dificultad de mantener y supervisar múltiples canales de venta y la necesidad de optimizar recursos para pequeños y medianos comercios que desean expandir su presencia en línea sin incurrir en complicaciones técnicas excesivas.

Desde su concepción, la plataforma ha sido diseñada con un enfoque claro en las necesidades de los pequeños y medianos negocios, que a menudo carecen de los recursos técnicos o humanos para gestionar eficientemente la presencia de su producto en línea. La simplicidad de uso, la automatización de procesos repetitivos y la accesibilidad de la aplicación buscan empoderar a este tipo de negocios, permitiéndoles competir en prácticamente igualdad de condiciones dentro del entorno del comercio electrónico global. De este modo, se contribuye a la digitalización del tejido comercial minorista, reduciendo barreras de entrada y fomentando su crecimiento sostenible.

Si bien es cierto que el proyecto se encuentra lejos de ser una solución completa, se ha tratado de aplicar buenas prácticas para garantizar su escalabilidad y su capacidad de adaptación a futuras necesidades. Se han implementado tecnologías modernas para asegurar una continuidad en el desarrollo del proyecto, permitiendo que se pueda seguir evolucionando y mejorando con el tiempo.

Ha sido un reto significativo, no solo por el desarrollo técnico de la aplicación, sino también por la gestión de un proyecto que en un futuro podría ser de gran envergadura. Es complicado detallar todo el proceso de desarrollo y las decisiones tomadas, pues ha sido un proceso iterativo y evolutivo, donde cada decisión ha influido en las siguientes. Sin embargo, se ha procurado documentar adecuadamente cada fase del proyecto para que pueda ser entendido y replicado en el futuro, hasta para aquellos que no tengan experiencia previa en el desarrollo de aplicaciones web.

En definitiva, este trabajo constituye un primer paso hacia una solución integral para la gestión multicanal de comercio electrónico, con la posibilidad de evolucionar y consolidarse como un posible negocio en el futuro. Los resultados obtenidos confirman que la idea inicial es viable y que la plataforma puede ser una herramienta valiosa para los pequeños y medianos comercios que buscan una alternativa para reducir la complejidad operativa resultado de su presencia en línea. Esta primera versión desarrollada sienta una base sólida para futuras mejoras, ampliaciones de funcionalidades y adaptaciones a las necesidades cambiantes del mercado.

\section{Trabajo futuro}

Aunque la plataforma desarrollada constituye una primera versión funcional, lejos se encuentra de ser una solución completa. Por ello, se han identificado diversas áreas de mejora y ampliación que podrían ser objeto de futuros trabajos. Algunas de las más relevantes son:

\begin{itemize}
    \item Implementación de un sistema de autenticación que permita a los usuarios tener su propia cuenta para así gestionar sus canales de venta.
    \item Implementación de un sistema de gestión de envíos más complejo, que permita a los usuarios crear envíos, tratar incidencias, crear etiquetas, entre muchos otros.
    \item Ampliación de la funcionalidad de análisis de ventas, permitiendo a los usuarios obtener informes más detallados y personalizados sobre sus ventas y productos.
    \item Integración con más plataformas de venta, ampliando así el número de canales que los usuarios pueden gestionar desde la aplicación.
    \item Implementación de una \gls{api} que permita a terceros desarrollar aplicaciones que se integren con la plataforma, pudiendo así vincular la aplicación con sistemas externos como \gls{crm} o sistemas de gestión de inventario.
    \item Implementar una funcionalidad de importación y exportación de datos, permitiendo a los usuarios subir productos desde archivos CSV o Excel, así como exportar sus datos a formatos comunes para su análisis externo.
\end{itemize}

Con todas estas implementaciones, se podría obtener una plataforma mucho más completa y funcional, lista para poder ser desplegada en un servidor y empezar a ser comercializada. A pesar de ser un proceso complejo y laborioso, la base sentada durante este proyecto es sólida y ha demostrado que la idea inicial es viable. Por lo tanto, el futuro del proyecto es prometedor y se espera que pueda seguir evolucionando y mejorando con el tiempo.