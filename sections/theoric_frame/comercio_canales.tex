\section{Comercio electrónico y canales de venta en línea}
\label{sec:comercio_canales}

En los últimos años, el comercio electrónico se ha convertido en una parte fundamental de la economía global. La capacidad de comprar y vender productos y servicios a través de internet ha transformado la forma en que las empresas interactúan con sus clientes. Este fenómeno ha dado lugar a la aparición de nuevos canales de venta, los llamados canales de venta en línea, que permiten a las empresas llegar a un público más amplio y diversificado, donde antes dependían de tiendas físicas o distribuidores locales.

No obstante, el comercio electrónico no es tan fácil como puede parecer en primera instancia. Existen tres grandes desafíos que las empresas deben enfrentar: la competencia, la infraestructura tecnológica y logística.

En internet todo el mundo juega con las mismas reglas; las facilidades que ofrece este medio son iguales para todos. El factor de proximidad al cliente ya no es el diferencial, sino la capacidad de ofrecer un producto o servicio que se diferencie del resto, tanto en calidad, precio o experiencia de compra. Esto genera que la competencia sea feroz, y las empresas deben encontrar formas innovadoras de destacar entre la multitud, tal como podrían ser las promociones, el marketing digital o la experiencia de usuario.

Por otro lado, el comercio electrónico requiere de una infraestructura tecnológica que permita listar productos, gestionar pedidos y pagos, y mantener una comunicación fluida con los clientes. Esto implica no solo contar con un sitio web atractivo y funcional, sino también con sistemas de gestión de inventario, plataformas de pago seguras y herramientas de análisis de datos que permitan tomar decisiones informadas. Todo esto puede resultar costoso y complicado de implementar, especialmente para pequeñas y medianas empresas que no cuentan con los recursos necesarios, ni en términos de personal, ni de dinero.

Por último, la logística es otro de los grandes retos del comercio electrónico. En el comercio tradicional, los productos se entregan directamente al cliente en la tienda. En el comercio electrónico, las empresas deben gestionar el almacenamiento, el envío y la entrega de productos a los clientes, lo que puede resultar complicado y costoso. La gestión de inventarios, la selección de proveedores de transporte y la coordinación de envíos son solo algunos de los aspectos logísticos que las empresas deben tener en cuenta para garantizar una experiencia de compra satisfactoria que cumpla con las expectativas de los clientes.

Estos tres factores son solo algunos de los muchos desafíos que enfrentan las empresas en el comercio electrónico. Por este mismo motivo, diferentes soluciones han surgido para ayudar a las empresas a superar estos obstáculos y aprovechar al máximo las oportunidades que ofrece el comercio en línea. Entre estas soluciones se encuentran dos que destacan por encima de las demás: las plataformas \textit{e-commerce} y los \textit{marketplaces}. Ambas ofrecen a las empresas la posibilidad de vender sus productos y servicios en línea, pero lo hacen de maneras diferentes.

\subsection{Plataformas \textit{E-commerce}}

Una plataforma \textit{e-commerce} \cite{adobe_ecommerce_platforms}

\subsection{\textit{Marketplaces}}

En este contexto, los marketplaces han emergido como una de las plataformas más populares para la venta en línea, ofreciendo a los vendedores la oportunidad de acceder a una base de clientes masiva sin necesidad de invertir en infraestructura propia.