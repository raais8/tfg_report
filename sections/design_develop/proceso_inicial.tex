\section{Proceso inicial de desarrollo de la plataforma}
\label{sec:proceso_desarrollo}

El desarrollo de la aplicación web no surge de simplemente decidir qué tecnologías se van a utilizar y empezar a programar. Antes de comenzar a desarrollar la plataforma se ha llevado a cabo un proceso de diseño que ha permitido definir la arquitectura del sistema, las tecnologías a utilizar y el flujo de trabajo.

\subsection{Separación de tecnologías \textit{frontend} y \textit{backend}}
\label{dev:subsec:separacion_frontend_backend}

El primer paso que se ha realizado y una vez ya definido el objetivo de la aplicación y las funcionalidades que se querían implementar, ha sido el de hacer un análisis de como estructurar la plataforma. Como ya se ha comentado en la sección \ref{sec:arquitectura_sistema}, se ha optado por una arquitectura dividida en dos partes: el \textit{backend}, incluyendo la base de datos, y el \textit{frontend}. No obstante, a pesar de que Django ofrece la posibilidad de crear ambas partes, se ha decidido utilizar React. Esta decisión ha supuesto un reto, ya que ha significado realizar un \textit{frontend} entero además de preparar una \gls{api} en el \textit{backend} para que ambos se puedan comunicar. Sin embargo, esta decisión ha permitido crear una aplicación más escalable, flexible y, sobre todo, dinámica.

Django es un \textit{framework} que funciona del lado del servidor, lo que significa que cada vez que se quiere mostrar una página distinta, el servidor tiene que procesar la petición y devolver la página completa. De esta manera, cuando el usuario cambia de página, el servidor carga todos los recursos (\gls{html}, \gls{css} y JavaScript) y los rellena con los datos necesarios, sirviendo una página estática. Por el contrario, React es un \textit{framework} que funciona del lado del cliente, lo que significa que el servidor solo tiene que enviar los datos necesarios y el cliente se encarga de maquetarlos y mostrarlos. Esto permite crear aplicaciones más dinámicas y rápidas, ya que no es necesario recargar la página cada vez que se quiere mostrar un nuevo contenido.

Este enfoque, a pesar de ser más complejo, es el estándar en la actualidad y es por este motivo que se ha optado por esta división de tecnologías.

\subsection{Herramientas de desarrollo y de control de versiones}
\label{dev:subsec:herramientas_desarrollo}

En un proyecto de desarrollo de software, es fundamental contar con herramientas que faciliten el trabajo y la gestión del código. Por este mismo motivo, se ha optado por hacer uso de dos herramientas muy comunes en el mundo del desarrollo: Visual Studio Code, Bruno y Git.

Visual Studio Code es un editor de código ampliamente utilizado por los desarrolladores, que ofrece una gran cantidad de extensiones y funcionalidades que facilitan el trabajo. Algunas de estas funcionalidades incluyen una terminal integrada, una vinculación con Git para el control de versiones, herramientas de depuración, entre muchas otras. Además, es un editor de código gratuito y de código abierto, lo que lo hace accesible para cualquier persona interesada en el desarrollo de software. De esta manera, es la herramienta que se ha utilizado para desarrollar tanto el \textit{frontend} como el \textit{backend} de la aplicación.

Bruno es una herramienta que permite probar y depurar \gls{api}s. Durante el desarrollo del \textit{backend}, se ha utilizado para probar las diferentes rutas de la \gls{api} y asegurarse de que funcionan correctamente. Bruno permite enviar peticiones HTTP a la \gls{api} y ver las respuestas, lo que facilita la depuración y el desarrollo de la misma. En este ámbito, existen herramientas más populares, como Postman, pero se ha optado por Bruno por ser de código abierto y por dar soporte a pequeños proyectos que hacen desarrolladores individuales o equipos pequeños sin esperar rédito económico alguno.

Por otro lado, Git es un sistema de control de versiones que permite llevar un seguimiento de los cambios realizados en el código a lo largo del tiempo. Esto es especialmente útil en proyectos de desarrollo, ya que permite revertir cambios, colaborar con otros desarrolladores y mantener un historial del proyecto. Git también es una herramienta gratuita y de código abierto, lo que la hace accesible para cualquier persona interesada en el desarrollo de software. A Git se le ha añadido una plataforma de alojamiento de código, GitHub, que permite almacenar el código en la nube y colaborar con otros desarrolladores. A pesar de que GitHub tiene sus planes de pago, se ha optado por utilizar la versión gratuita, que permite almacenar proyectos académicos y personales de manera ilimitada.

Por último, resulta imprescindible mencionar una herramienta que en los últimos años se ha vuelto indispensable en numerosos ámbitos: los agentes de inteligencia artificial. En este proyecto, se ha empleado ChatGPT como apoyo para resolver dudas concretas, clarificar conceptos complejos y obtener ejemplos de código cuando la documentación oficial no resultaba suficiente o no se encontraba la información deseada. No obstante, es importante destacar que, tratándose de un proyecto de esta envergadura, donde se superan las 9000 líneas de código, no se puede esperar que una IA realice por sí sola todo el trabajo necesario. Esta herramienta actúa como un recurso que potencia y agiliza el desarrollo, pero en ningún caso sustituye al desarrollador ni elimina la necesidad de comprender a fondo las tecnologías utilizadas. Bien empleada, la inteligencia artificial complementa la labor humana, permitiendo optimizar tiempos y facilitar la resolución de problemas puntuales, pero siempre bajo la supervisión y el criterio del desarrollador.

Así pues, con todas las herramientas definidas, a continuación se encuentra el enlace al repositorio del proyecto, donde se puede encontrar todo el código fuente de la aplicación, tanto del \textit{backend} como del \textit{frontend}. El repositorio está alojado en GitHub y es de acceso público, por lo que cualquier persona interesada puede consultarlo y contribuir al proyecto.

\begin{center}
    \boxed{\href{https://github.com/raais8/mpmanager_back}{\text{Repositorio \textit{Backend}}}}
    \hspace{1cm}
    \boxed{\href{https://github.com/raais8/mpmanager_front}{\text{Repositorio \textit{Frontend}}}}
\end{center}