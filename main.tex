\documentclass[12pt]{report}

\usepackage[utf8]{inputenc}
\usepackage{graphicx}
\usepackage{geometry}
\usepackage{vmargin}
\usepackage{float}
\usepackage{wrapfig}
\usepackage[dvipsnames]{xcolor}
\usepackage{fancyhdr}
\usepackage[acronym,nomain,nonumberlist]{glossaries}
\usepackage{setspace}
\usepackage{hyperref}
\usepackage[spanish]{babel}
\usepackage[backend=biber,style=iso-numeric]{biblatex}
\addbibresource{refs.bib}


% Document margins
\setpapersize{A4}
\setmargins{2.2cm}          % margen izquierdo
{0cm}                       % margen superior
{16.5cm}                    % anchura del texto
{24.57cm}                   % altura del texto
{55pt}                      % altura cabeceras
{1.25cm}                    % espacio entre el texto y los cabeceras
{1pt}                       % altura del pie de página
{1cm}                       % Espacio entre el texto y el pie de página

% Line spacing
\renewcommand{\baselinestretch}{1.5} 

% Remove paragraph indentation
\setlength{\parindent}{0pt}

% Paragraph spacing
\setlength{\parskip}{1em}

% Definir colores personalizados
\definecolor{linkcolor}{RGB}{52, 152, 219}   % Azul claro
\definecolor{citecolor}{RGB}{231, 76, 60}    % Rojo suave
\definecolor{urlcolor}{RGB}{46, 204, 113}    % Verde esmeralda
\definecolor{filecolor}{RGB}{155, 89, 182}   % Morado
\definecolor{menucolor}{RGB}{241, 196, 15}   % Amarillo dorado
\definecolor{codegray}{gray}{0.95}           % Gris claro para el código

% Configuración de Hyperref con colores
\hypersetup{
    colorlinks=true,
    linkcolor=linkcolor,
    citecolor=citecolor,
    urlcolor=linkcolor,
    filecolor=filecolor,
    menucolor=menucolor,
    bookmarksopen=true,
    bookmarksnumbered=true,
    pdfstartview=FitH,
    pdfpagemode=UseOutlines
}

% Configuración de los fragmentos de código
\lstdefinestyle{mystyle}{
    backgroundcolor=\color{white},       % Fondo blanco
    basicstyle=\ttfamily\footnotesize,   % Fuente monoespaciada y pequeña
    keywordstyle=\color{RoyalBlue},      % Palabras clave en azul
    commentstyle=\color{ForestGreen},    % Comentarios en verde oscuro
    stringstyle=\color{BrickRed},        % Strings en rojo ladrillo
    numberstyle=\tiny\color{gray},       % Números de línea en gris
    frame=tb,                            % Marco arriba y abajo
    rulecolor=\color{gray},              % Color del marco
    aboveskip=3mm,
    belowskip=3mm,
    tabsize=4,
    showstringspaces=false,
    showspaces=false,
    showtabs=false,
    breaklines=true,
    breakatwhitespace=false,
    escapeinside={\%*}{*)},
    keepspaces=true,
    captionpos=b,
    framexleftmargin=16pt,
    framextopmargin=3pt,
    framexbottommargin=6pt,
    extendedchars=true,
    title=\lstname
}

\lstset{style=mystyle}

\colorlet{punct}{red!60!black}
\definecolor{background}{HTML}{EEEEEE}
\definecolor{delim}{RGB}{20,105,176}
\colorlet{numb}{magenta!60!black}

\lstdefinelanguage{json}{
    stepnumber=1,
    numbersep=8pt,
    showstringspaces=false,
    breaklines=true,
    frame=lines,
    backgroundcolor=\color{white},
    aboveskip=0mm,
    belowskip=0mm,
    literate=
     *{0}{{{\color{numb}0}}}{1}
      {1}{{{\color{numb}1}}}{1}
      {2}{{{\color{numb}2}}}{1}
      {3}{{{\color{numb}3}}}{1}
      {4}{{{\color{numb}4}}}{1}
      {5}{{{\color{numb}5}}}{1}
      {6}{{{\color{numb}6}}}{1}
      {7}{{{\color{numb}7}}}{1}
      {8}{{{\color{numb}8}}}{1}
      {9}{{{\color{numb}9}}}{1}
      {:}{{{\color{punct}{:}}}}{1}
      {,}{{{\color{punct}{,}}}}{1}
      {\{}{{{\color{delim}{\{}}}}{1}
      {\}}{{{\color{delim}{\}}}}}{1}
      {[}{{{\color{delim}{[}}}}{1}
      {]}{{{\color{delim}{]}}}}{1},
}

% Reducir el espacio entre líneas en listas
\setlist[itemize]{nosep, topsep=0pt, partopsep=0pt}

% Header  configuration
\pagestyle{fancy}
\fancypagestyle{plain}{
    \fancyhf{}
    \rhead{{Desarrollo de una aplicación web para la gestión \\ centralizada de múltiples marketplaces}}
    \lhead{\includegraphics[width=5cm]{figures/generic/logo_eseiaat.png}}
    \cfoot{\thepage}
}
\fancyhf{}
\rhead{{Desarrollo de una aplicación web para la gestión \\ centralizada de múltiples marketplaces}}
\lhead{\includegraphics[width=5cm]{figures/generic/logo_eseiaat.png}}
\cfoot{\thepage}

% Create acronym list
\makenoidxglossaries
\newacronym{saas}{SaaS}{Software as a Service}
\newacronym{api}{API}{Application Programming Interface}
\newacronym{crm}{CRM}{Customer Relationship Management}

\begin{document}

% Rename of variables
\renewcommand{\listfigurename}{Índice de figuras}
\renewcommand{\listtablename}{Índice de tablas}
\renewcommand{\contentsname}{Sumario}
\renewcommand{\figurename}{Figura}
\renewcommand{\tablename}{Tabla}
\renewcommand{\chaptername}{Capítulo}
\renewcommand{\bibname}{Referencias}
\renewcommand*{\acronymname}{Acrónimos}
\renewcommand{\lstlistingname}{Fragmento}


\pagenumbering{gobble}
\newgeometry{left=5cm,bottom=0.1cm,textwidth=19cm}

{
    \setstretch{1}
    \parskip=0pt
    \fontsize{10pt}{12pt}

    \pagestyle{empty}

    \begin{figure}[h]
        \includegraphics[width=6cm]{figures/generic/logo_eseiaat.png}
    \end{figure}

    \begin{wrapfigure}{r}{0.25\textwidth}
        \centering
        \includegraphics[angle=90, width=0.07\textwidth]{figures/generic/Trabajofindeestudios.jpg}
    \end{wrapfigure}

    \phantom{abs}

    \vspace{1.5cm}

    \textbf{\scalebox{2.5}{\textcolor{RoyalBlue}{\parbox{\textwidth}{Desarrollo de una aplicación web \\ para la gestión centralizada de \\ múltiples marketplaces}}}}

    \vspace{2cm}

    \textbf{\scalebox{1.5}{{Documento:}}}

    \vspace{0.5 cm}
    \scalebox{1.5}{\textcolor{RoyalBlue}{Memoria}}

    \vspace{2cm}

    \textbf{\scalebox{1.5}{{Autor:}}}

    \vspace{0.5 cm}
    \scalebox{1.5}{{\textcolor{RoyalBlue}{Aleix Ribas Torras}}}

    \vspace{2cm}

    \textbf{\scalebox{1.5}{{Director - Codirectora:}}}

    \vspace{0.5 cm}
    \scalebox{1.5}{\textcolor{RoyalBlue}{\parbox{\textwidth}{Francisco José Múgica Alvarez \\ Maria Angela Nebot Castells}}}

    \vspace{2cm}

    \textbf{\scalebox{1.5}{{Titulación:}}}

    \vspace{0.5 cm}
    \scalebox{1.5}{{\textcolor{RoyalBlue}{Grado en Ingeniería en Tecnologías Aeroespaciales}}}

    \vspace{2cm}

    \textbf{\scalebox{1.5}{{Convocatoria:}}}

    \vspace{0.5 cm}

    \scalebox{1.5}{{\textcolor{RoyalBlue}{Primavera, 2025}}}

}

\restoregeometry

\newpage
\thispagestyle{empty}
\phantom{abc}
\newpage

\pagenumbering{Roman}

\chapter*{Resumen}
El presente proyecto desarrolla una aplicación web para la gestión centralizada de múltiples marketplaces, concebida como una potente herramienta de automatización y un método unificado para optimizar las operaciones de comercio electrónico. El sector del e-commerce, en constante crecimiento, enfrenta desafíos significativos como la feroz competencia, la complejidad de la infraestructura tecnológica y logística, y los altos costos asociados a la administración de productos, pedidos y pagos en diversos canales de venta, lo que tradicionalmente consume valiosos recursos y tiempo.

Para superar estos obstáculos y aprovechar las oportunidades del comercio en línea, la solución propuesta actúa como un método estratégico y centralizado para administrar eficientemente los canales de venta. Su objetivo principal es centralizar la información y la gestión de pedidos y productos de diferentes marketplaces en una única plataforma, ofreciendo una visión unificada del negocio.

Esta solución representa una primera versión funcional que sienta las bases para lo que, en el futuro, podría evolucionar hacia una plataforma integral de gestión multicanal, capaz de escalar, adaptarse a nuevos marketplaces y ofrecer herramientas avanzadas de análisis, automatización y toma de decisiones para los comercios electrónicos.

\chapter*{Abstract}
This project develops a web application for the centralized management of multiple marketplaces, conceived as a powerful automation tool and a unified method to optimize e-commerce operations. The e-commerce sector, which is constantly growing, faces significant challenges such as an strong competition, the complexity of technological and logistical infrastructure, and the high costs associated with managing products, orders, and payments across various sales channels, raditionally consuming valuable time and resources.

To overcome these obstacles and seize the opportunities of online commerce, the proposed solution acts as a strategic and centralized method for efficiently managing sales channels. Its main goal is to centralize information and the management of orders and products from different marketplaces into a single platform, offering a unified view of the business.

This solution represents a first functional version that lays the groundwork for what could, in the future, evolve into a comprehensive multichannel management platform, capable of scaling, adapting to new marketplaces, and providing advanced tools for analysis, automation, and decision-making for e-commerce businesses.

\selectlanguage{spanish}

\tableofcontents
\listoffigures
\glsaddall
\renewcommand{\thepage}{\Roman{page}}
\printnoidxglossaries

\newpage
\pagenumbering{arabic}
\renewcommand{\thepage}{\arabic{page}}

\chapter{Introducción}
\label{chap:intro}
\chapter{Marco Teórico}
\label{chap:marco_teorico}

En este capítulo se presenta el marco teórico que sustenta el desarrollo de la aplicación web para la gestión centralizada de múltiples \textit{marketplaces}. Se abordan conceptos clave relacionados con el comercio en línea, sus dificultades y sus múltiples vertientes. Además, se exploran las tecnologías y herramientas utilizadas en el desarrollo de la aplicación, así como las metodologías de trabajo adoptadas durante el proceso.

\section{Comercio electrónico y canales de venta en línea}
\label{sec:comercio_canales}

En los últimos años, el comercio electrónico se ha convertido en una parte fundamental de la economía global. La capacidad de comprar y vender productos y servicios a través de internet ha transformado la forma en que las empresas interactúan con sus clientes. Este fenómeno ha dado lugar a la aparición de nuevos canales de venta, los llamados canales de venta en línea, que permiten a las empresas llegar a un público más amplio y diversificado, donde antes dependían de tiendas físicas o distribuidores locales.

No obstante, el comercio electrónico no es tan fácil como puede parecer en primera instancia. Existen tres grandes desafíos que las empresas deben enfrentar: la competencia, la infraestructura tecnológica y logística.

En internet todo el mundo juega con las mismas reglas; las facilidades que ofrece este medio son iguales para todos. El factor de proximidad al cliente ya no es el diferencial, sino la capacidad de ofrecer un producto o servicio que se diferencie del resto, tanto en calidad, precio o experiencia de compra. Esto genera que la competencia sea feroz, y las empresas deben encontrar formas innovadoras de destacar entre la multitud, tal como podrían ser las promociones, el marketing digital o la experiencia de usuario.

Por otro lado, el comercio electrónico requiere de una infraestructura tecnológica que permita listar productos, gestionar pedidos y pagos, y mantener una comunicación fluida con los clientes. Esto implica no solo contar con un sitio web atractivo y funcional, sino también con sistemas de gestión de inventario, plataformas de pago seguras y herramientas de análisis de datos que permitan tomar decisiones informadas. Todo esto puede resultar costoso y complicado de implementar, especialmente para pequeñas y medianas empresas que no cuentan con los recursos necesarios, ni en términos de personal, ni de dinero.

Por último, la logística es otro de los grandes retos del comercio electrónico. En el comercio tradicional, los productos se entregan directamente al cliente en la tienda. En el comercio electrónico, las empresas deben gestionar el almacenamiento, el envío y la entrega de productos a los clientes, lo que puede resultar complicado y costoso. La gestión de inventarios, la selección de proveedores de transporte y la coordinación de envíos son solo algunos de los aspectos logísticos que las empresas deben tener en cuenta para garantizar una experiencia de compra satisfactoria que cumpla con las expectativas de los clientes.

Estos tres factores son solo algunos de los muchos desafíos que enfrentan las empresas en el comercio electrónico. Por este mismo motivo, diferentes soluciones han surgido para ayudar a las empresas a superar estos obstáculos y aprovechar al máximo las oportunidades que ofrece el comercio en línea. Entre estas soluciones se encuentran dos que destacan por encima de las demás: las plataformas \textit{e-commerce} y los \textit{marketplaces}. Ambas ofrecen a las empresas la posibilidad de vender sus productos y servicios en línea, pero lo hacen de maneras diferentes.

\subsection{Plataformas \textit{E-commerce}}

Una plataforma \textit{e-commerce} \cite{adobe_ecommerce_platforms}

\subsection{\textit{Marketplaces}}

En este contexto, los marketplaces han emergido como una de las plataformas más populares para la venta en línea, ofreciendo a los vendedores la oportunidad de acceder a una base de clientes masiva sin necesidad de invertir en infraestructura propia.

% \section{Modelos de distribución de software SaaS}
\label{sec:modelos_negocio}

Hay una gran variedad de modelos de distribución de software, tal como pueden ser el modelo \textit{On-Premise}, el modelo \textit{Infrastructure as a Service} (IaaS) o el modelo \textit{Platform as a Service} (PaaS). Sin embargo, el modelo que más se utiliza en la actualidad es el modelo \textit{Software as a Service} (SaaS).

Años atrás, el software se distribuía principalmente a través de licencias perpetuas, donde los usuarios compraban una licencia para utilizar el software en sus propios servidores o computadoras. Este modelo requería que los usuarios gestionaran la infraestructura y el mantenimiento del software, lo que podía ser costoso y complicado. Esto significaba que el proveedor del software simplemente facilitaba el producto y el usuario debía hacerse cargo de la instalación, configuración y mantenimiento del mismo. Esto podía resultar complicado y costoso, especialmente para pequeñas y medianas empresas que no contaban con los recursos necesarios para gestionar su propia infraestructura. Esto es conocido como una infraestructura \textit{On-Premise}.

Con la llegada de internet y la nube, surgieron nuevos modelos de distribución de software que permitieron a las empresas ofrecer sus productos y servicios de manera más eficiente y escalable. El modelo SaaS es uno de los más populares y se basa en la idea de que el software se aloja en la nube y se accede a través de internet. Esto significa que los usuarios no necesitan instalar ni gestionar el software en sus propios servidores u ordenadores, sino que pueden acceder a él a través de un navegador web.

Sin embargo, la dependencia de la conexión a internet, la falta de control sobre la infraestructura y la seguridad de los datos son puntos críticos en el modelo. Además, los proveedores de SaaS suelen cobrar tarifas mensuales o anuales por el uso del software, pues el mantenimiento de la infraestructura y el soporte técnico son responsabilidad del proveedor.

Existen también otras alternativas, como pueden ser los modelos IaaS y PaaS. Cada uno de estos modelos tiene sus propias ventajas y desventajas, y dependiendo del tipo de negocio y las necesidades del cliente, uno puede ser más adecuado que otro. En la figura \ref{fig:modelos_negocio} se pueden observar los diferentes modelos de distribución de software y sus características.

\begin{figure}
    \centering
    \includegraphics[width=0.8\textwidth]{figures/theoric_frame/comp_services.png}
    \caption{Modelos de distribución de software. Fuente: \cite{modelos_distirbucion}}
    \label{fig:modelos_negocio}
\end{figure}

\section{Arquitectura y tecnologías de una aplicación web}
\label{sec:arquitectura_sistema}

En el mundo del software existen dos tipos de aplicaciones: las aplicaciones de escritorio y las aplicaciones web. Las aplicaciones de escritorio son aquellas que se instalan en un ordenador y se ejecutan de forma local, mientras que las aplicaciones web son aquellas que se ejecutan en un servidor y se acceden a través de un navegador web.

Desde los inicios del desarrollo de software, las aplicaciones de escritorio han sido la norma. Sin embargo, en los últimos años ha habido un cambio significativo hacia el desarrollo de aplicaciones web, pues el avance de las distintas tecnologías web y la conectividad a internet han mitigado considerablemente las desventajas que anteriormente presentaban \cite{evo_web}.

Las aplicaciones de escritorio no requieren de un servidor externo para funcionar, toda la lógica se encuentra en el ordenador del usuario y es este mismo quien lo ejecuta. Esto hace que la aplicación sea más rápida y eficiente, ya que no hay necesidad de enviar datos a través de internet. No obstante, esto también significa que el usuario debe instalar la aplicación en su ordenador y mantenerla actualizada, lo que puede ser un inconveniente \cite{web_vs_desktop}.

Por otro lado, las aplicaciones web sí que requieren de un servidor externo para funcionar. Esto significa que el usuario no necesita instalar nada en su ordenador, ya que la aplicación se ejecuta en el servidor y se accede a través de un navegador web. Este paradigma de desarrollo permite que la aplicación sea más accesible, pues se puede acceder a la aplicación desde cualquier dispositivo y lugar con conexión a internet. Además, como todo se encuentra en el servidor y no en el usuario, las actualizaciones son mucho más inmediatas. Sin embargo, esto también significa que la aplicación puede ser más lenta y menos eficiente debido a que la lógica y los datos no se encuentran en el ordenador del usuario, sino en el servidor \cite{web_vs_desktop}.

La mejor conectividad a internet y el avance de las tecnologías web han permitido que las aplicaciones web sean cada vez más rápidas y eficientes. Esto ha llevado a un aumento en la popularidad de las aplicaciones web, y muchas empresas están optando por desarrollar aplicaciones web en lugar de aplicaciones de escritorio.

Con todo esto, se ha formado lo que se conoce como arquitectura web, que hace referencia a la forma en que se organiza y estructura el software. Esto incluye la forma en que se comunican los diferentes componentes de la aplicación, así como la forma en que se almacenan y gestionan los datos. Todas las aplicaciones web están compuestas por tres componentes principales: el cliente, la lógica de negocio y la base de datos.

Estos tres componentes se dividen en dos bloques: el \textit{frontend} y el \textit{backend}. El \textit{frontend} es la parte de la aplicación que interactúa con el usuario, es decir, el cliente, mientras que el \textit{backend} es la parte de la aplicación que se encarga de gestionar los datos y la lógica de negocio. Ambos bloques se comunican entre sí a través de una API (Interfaz de Programación de Aplicaciones), que es un conjunto de reglas y protocolos que permiten que diferentes componentes de software se comuniquen entre sí. Todo esto se puede ver en la figura \ref{fig:arquitectura_web}.

\begin{figure}
    \centering
    \includegraphics[width=0.8\textwidth]{figures/theoric_frame/arquitectura_web.pdf}
    \caption{Arquitectura de una aplicación web.}
    \label{fig:arquitectura_web}
\end{figure}

\subsection{\textit{Frontend}}

El \textit{frontend} es la parte de la aplicación que interactúa con el usuario. Esto incluye la interfaz de usuario, que es la parte de la aplicación que el usuario ve y con la que interactúa, así como la lógica de presentación, que es la parte de la aplicación que se encarga de mostrar los datos al usuario. El \textit{frontend} se desarrolla principalmente utilizando HTML, CSS y JavaScript, tecnologías que se ejecutan de manera nativa en los navegadores.

El HTML (\textit{Hypertext Markup Language}) es el lenguaje de marcado utilizado para estructurar el contenido de una página web. El CSS (\textit{Cascading Style Sheets}) es el lenguaje utilizado para dar estilo a una página web, es decir, para definir cómo se verá el contenido estructurado por el HTML. Por último, JavaScript es un lenguaje de programación que se utiliza para añadir interactividad a una página web, es decir, para permitir que el usuario interactúe con la aplicación.

Con todo esto, el servidor envía todo este contenido al usuario de manera que su navegador lo pueda representar, en caso del HTML y el CSS, y ejecutar, en caso del JavaScript.

Desarrollar aplicaciones de manera nativa, es decir, utilizando HTML, CSS y JavaScript sin el apoyo de ninguna librería o herramienta adicional, puede ser complicado, tedioso y poco eficiente. Para facilitar esta tarea, existen distintos \textit{frameworks} que proporcionan un conjunto de herramientas y funcionalidades pensadas para abstraer la complejidad del desarrollo. De esta manera, los \textit{frameworks} cumplen principalmente dos propósitos:

\begin{itemize}
    \item \textbf{Ahorrar tiempo:} Los \textit{frameworks} permiten al desarrollador ahorrar tiempo ofreciendo funcionalidades predefinidas a problemas comunes o recurrentes. En el caso de una aplicación web, esto puede incluir la gestión de rutas, la gestión del estado de la aplicación, la gestión de formularios, entre otros \cite{framworks}.
    \item \textbf{Garantizar buenas prácticas de desarrollo:} Los \textit{frameworks} suelen seguir patrones de diseño y buenas prácticas de desarrollo que ayudan a los desarrolladores a escribir código limpio, mantenible y escalable. Adicionalmente ofrecen características de seguridad por defecto, de manera que el desarrollador no tiene que implementar sus propias medidas de seguridad que pueden resultar ser vulnerables. Un ejemplo es la autenticación de usuarios, donde el \textit{framework} se encarga de gestionar la creación y validación de los \textit{tokens} de acceso, así como la gestión de sesiones \cite{framworks}.
\end{itemize}

Algunos de los \textit{frameworks} más populares para el desarrollo de \textit{frontend} son React, Angular y Vue.js. En el caso de este proyecto, se ha optado por utilizar React, un \textit{framework} desarrollado por Facebook que en los últimos años se ha convertido en un estándar de la industria. React es un \textit{framework} basado en componentes, lo que significa que la interfaz de usuario se divide en unidades independientes y reutilizables, facilitando así la creación de aplicaciones más complejas y escalables.

React se desarrolla utilizando JavaScript, un lenguaje de programación de tipado dinámico, lo que significa que no requiere de declarar el tipo de las variables al momento de crearlas. Esta característica, si bien ofrece flexibilidad, puede provocar errores difíciles de detectar. Para solventarlo, existe TypeScript, un lenguaje que extiende JavaScript añadiendo tipado estático \cite{typescript}. Con TypeScript, el desarrollador puede especificar el tipo de las variables, permitiendo que los errores de tipado se detecten en tiempo de interpretación. Aunque los navegadores solo interpretan JavaScript, el código en TypeScript se transpila automáticamente a JavaScript. Por este motivo, en este proyecto se ha decidido utilizar TypeScript para mejorar la calidad y la robustez del código \cite{tipado}.

Finalmente, otro motivo relevante para la elección de React es su gran comunidad de desarrolladores, así como la amplia disponibilidad de librerías y herramientas que extienden sus funcionalidades básicas. La elección de React y el detalle de su funcionamiento se explican en profundidad en la sección \textcolor{red}{(Añadir sección parte desarrollo frontend)}.

\subsection{\textit{Backend}}

El \textit{backend} es la parte de la aplicación que administra la funcionalidad general de la aplicación. Cuando el usuario interactúa con el \textit{frontend}, la interacción envía una solicitud al \textit{backend} para que la procese y devuelva una respuesta \cite{aws_frontend_backend}. De este modo, el \textit{backend} se encarga de gestionar la lógica de negocio, es decir, es la parte de la aplicación que se encarga de procesar los datos y realizar las operaciones necesarias para realizar las distintas funcionalidades de ésta. Adicionalmente contiene la base de datos, que es donde se almacenan todos los datos de la aplicación.

A diferencia del \textit{frontend}, el \textit{backend} no se ejecuta en el navegador del usuario, sino en un servidor. Esto significa que el \textit{backend} puede utilizar lenguajes de programación y tecnologías que no son compatibles con los navegadores, como Java, Python o Ruby. No obstante, un factor que si tiene en común con el \textit{frontend} es la existencia de \textit{frameworks}.

Para el desarrollo de este proyecto se ha decidido utilizar Django, un \textit{framework} escrito en Python. Django es un \textit{framework} un tanto especial, ya que permite desarrollar tanto el \textit{frontend} como el \textit{backend} de una aplicación web. Sin embargo, en este proyecto se ha optado por utilizar Django únicamente para el desarrollo del \textit{backend}, dejando el \textit{frontend} a cargo de React, pues se considera que es la mejor opción para el desarrollo de aplicaciones web modernas. En cuanto al aspecto más técnico, Django está basado en el modelo MVC (Modelo-Vista-Controlador), un patrón de diseño que separa la lógica en tres componentes principales: el modelo, que se encarga de gestionar los datos; la vista, que se encarga de mostrar los datos al usuario; y el controlador, que se encarga de gestionar la interacción entre el modelo y la vista \cite{mvc}.

El funcionamiento del modelo MVC se representa en la figura \ref{fig:mvc}. Para entender mejor este flujo, se puede tomar como ejemplo una acción concreta dentro del proyecto: acceder a la sección de productos de la aplicación. A continuación, se explica paso a paso lo que ocurre en ese proceso:

\begin{enumerate}
    \item \textbf{Solicitud del usuario} El usuario accede a la sección de productos desde el \textit{frontend}, ya sea a través de un menú o directamente introduciendo una URL. Esta acción genera una solicitud HTTP que se envía a un \textit{endpoint} del \textit{backend} (paso 1 en la figura).
    \item \textbf{Recepción por parte del controlador:} El \textit{endpoint} está vinculado a una función del controlador, que es el encargado de procesar la solicitud. En este caso, el controlador interpreta que se necesita acceder a los datos de los productos y actúa en consecuencia (paso 2).
    \item \textbf{Consulta al modelo:} El controlador solicita la información al modelo correspondiente, es decir, a los productos. El modelo representa la estructura de datos y contiene la lógica necesaria para interactuar con la base de datos de manera más sencilla y segura (paso 3).
    \item \textbf{Respuesta del modelo:} El modelo realiza la consulta a la base de datos y devuelve al controlador los datos solicitados, en este caso, la lista de productos (paso 4).
    \item \textbf{Selección de la vista:} Con los datos recibidos, el controlador selecciona la vista adecuada para estructurar la respuesta. Esta vista se encarga de preparar los datos en un formato que el \textit{frontend} pueda interpretar, normalmente JSON (\textit{JavaScript Object Notation}) (paso 5).
    \item \textbf{Respuesta al usuario:} La vista estructurada en formato JSON se devuelve al controlador, que finalmente la envía como respuesta al usuario. El \textit{frontend} recibe estos datos y se encarga de representarlos en la interfaz gráfica, mostrando al usuario la información solicitada: los productos (paso 6).
\end{enumerate}

\begin{figure}
    \centering
    \includegraphics[width=0.8\textwidth]{figures/theoric_frame/mvc.pdf}
    \caption{Diagrama de flujo del patrón de diseño MVC.}
    \label{fig:mvc}
\end{figure}

Una mayor profundidad sobre el funcionamiento de los modelos y la base de datos se puede encontrar en la sección \ref{sec:base_datos} y sobre la comunicación entre el \textit{frontend} y el \textit{backend} en la sección \ref{sec:api}.

\subsection{Base de datos}
\label{sec:base_datos}

\subsection{API}
\label{sec:api}

\begin{figure}
    \centering
    \includegraphics[width=0.8\textwidth]{figures/theoric_frame/use_case.pdf}
    \caption{Diagrama de flujo del proceso de autenticación de usuario como caso de uso.}
    \label{fig:use_case}
\end{figure}


\chapter{Diseño y Desarrollo de la plataforma}
\label{chap:diseno_desarrollo}

En este capítulo se explicará todo el proceso de diseño y desarrollo de la plataforma, dando especial énfasis a la justificación de las decisiones tomadas y a la explicación de los distintos problemas que se han ido encontrando a lo largo del proceso. En concreto, se detallará el diseño de la base de datos, el desarrollo del \textit{backend} y el desarrollo del \textit{frontend}. No obstante, antes de entrar en detalle en cada una de estas secciones, se explicará el proceso inicial de desarrollo de la plataforma y se justificarán las tecnologías elegidas, cumplimentando así la sección \ref{sec:arquitectura_sistema} del capítulo \ref{chap:marco_teorico}.

% \textcolor{red}{Explicar que s'ha decidit fer un saas i una app web degut a les aventatges anteriorment esmentades.}

% \textcolor{red}{Ja s'ha dit que es farà servir React i Django i s'ha explicat breument com funcionen. Afegir el perquè d'aquestes eleccions.}

\section{Proceso inicial de desarrollo de la plataforma}
\label{sec:proceso_desarrollo}

El desarrollo de la aplicación web no surge de simplemente decidir qué tecnologías se van a utilizar y empezar a programar. Antes de comenzar a desarrollar la plataforma se ha llevado a cabo un proceso de diseño que ha permitido definir la arquitectura del sistema, las tecnologías a utilizar y el flujo de trabajo.

\subsection{Separación de tecnologías \textit{frontend} y \textit{backend}}

El primer paso que se ha realizado y una vez ya definido el objetivo de la aplicación y las funcionalidades que se querían implementar, se ha llevado a cabo un análisis de como estructurar la plataforma. Como ya se ha comentado en la sección \ref{sec:arquitectura_sistema}, se ha optado por una arquitectura dividida en dos partes: el \textit{backend}, incluyendo la base de datos, y el \textit{frontend}. No obstante, a pesar de que Django ofrece la posibilidad de crear ambas partes, se ha decidido utilizar React. Esta decisión ha supuesto un reto, ya que ha significado realizar un \textit{frontend} entero además de preparar una API en el \textit{backend} para que ambos se puedan comunicar. Sin embargo, esta decisión ha permitido crear una aplicación más escalable, flexible y, sobre todo, dinámica.

Django es un \textit{framework} que funciona del lado del servidor, lo que significa que cada vez que se quiere mostrar una página distinta, el servidor tiene que procesar la petición y devolver la página completa. De esta manera, cuando el usuario cambia de página, el servidor carga todos los recursos (HTML, CSS y JavaScript) y los rellena con los datos necesarios, sirviendo una página estática. Por el contrario, React es un \textit{framework} que funciona del lado del cliente, lo que significa que el servidor solo tiene que enviar los datos necesarios y el cliente se encarga de mostrar la información. Con esto, el servidor solo tiene que enviar los datos necesarios y el cliente se encarga de mostrar la información. Esto permite crear aplicaciones más dinámicas y rápidas, ya que no es necesario recargar la página cada vez que se quiere mostrar un nuevo contenido.

Este enfoque, a pesar de ser más complejo, es el estándar en la actualidad y es por este motivo que se ha optado por esta división de tecnologías.

\section{Diseño de la base de datos}
\label{sec:diseno_base_datos}

Para estructurar el proyecto, se ha optado por empezar definiendo la base de datos. Diseñar la base de datos inicialmente permite tener una visión general de como se va a estructurar el proyecto y como sus distintas partes se van a relacionar entre sí. Al fin y al cabo, la base de datos es el núcleo de la aplicación, ya que de ella dependen todas las funcionalidades.

\subsection{Bloques de funcionalidades}

Antes de empezar a diseñar la base de datos, se deben definir las funcionalidades claves de la aplicación para así poder estructurar los datos de manera que se puedan implementar de la mejor manera posible. En este caso, las funcionalidades claves son las siguientes:

\begin{itemize}
    \item \textbf{Gestión de pedidos:} La herramienta debe centralizar todos los pedidos de los distintos canales de venta en línea y permitir la gestión de los mismos. Esto incluye la posibilidad de crear, editar y eliminar pedidos, así como la posibilidad de marcar un pedido como enviado o entregado.
    \item \textbf{Gestión de productos:} La herramienta debe permitir la gestión de los productos disponibles en los distintos canales de venta en línea. Esto incluye la posibilidad de crear, editar y eliminar productos de los distintos canales, así como la posibilidad de editar sus atributos, tales como el precio, la descripción y la imagen, entre muchos otros.
    \item \textbf{Gestión de canales:} La herramienta debe permitir la gestión de los distintos canales de venta en línea. Esto incluye la posibilidad de crear, editar y eliminar canales.
    \item \textbf{Gestión de usuarios:} La herramienta debe permitir la gestión de los usuarios que pueden acceder a la aplicación. Esto incluye la posibilidad de crear, editar y eliminar usuarios, así como la posibilidad de asignarles distintos permisos y roles dentro de la aplicación.
\end{itemize}

Con las tres funcionalidades clave definidas, se puede concluir que la base de datos debe contener tres tablas principales: una para los pedidos, otra para los productos, otra para los canales y una última para los usuarios. A partir de aquí, se pueden definir las distintas tablas que van a complementar las principales.

\subsubsection{Bloque de pedidos}

El bloque de pedidos es el conjunto de tablas y relaciones que almacenan toda la información correspondiente a los pedidos. En cada pedido es importante almacenar la información de éste, como el estado, la fecha, el método de pago, el canal de venta, entre otros. Además, para saber donde se debe enviar el pedido, es importante almacenar la información del cliente, como su nombre, dirección y teléfono. Por último, también es importante almacenar la información de los productos que componen el pedido, como su nombre, precio y cantidad solicitada.

Conociendo la información que se debe almacenar, se pueden definir las siguientes tablas:

\begin{itemize}
    \item \textbf{Pedido [\texttt{order}]:} Esta tabla almacena la información general de cada pedido. Los campos que contiene son los siguientes:
          %   \begin{itemize}
          %       \item \texttt{id}: Identificador único del pedido. \textit{Clave primaria (entero)}.
          %       \item \texttt{order\_id}: Identificador del pedido en el canal de venta. \textit{Cadena de caracteres}.
          %       \item \texttt{status}: Estado del pedido (pendiente, enviado, entregado, cancelado). \textit{Entero}.
          %       \item \texttt{order\_date}: Fecha en la que se realizó el pedido. \textit{Fecha y hora}.
          %       \item \texttt{total\_price}: Precio total del pedido. \textit{Decimal}.
          %       \item \texttt{ticket}: Número de ticket del pedido. \textit{Cadena de caracteres}.
          %       \item \texttt{ticket\_refund}: Número de ticket de la devolución del pedido. \textit{Cadena de caracteres}.
          %       \item \texttt{pay\_method}: Método de pago del pedido (tarjeta, transferencia, efectivo). \textit{Entero}.
          %       \item \texttt{package\_quantity}: Cantidad de bultos (paquetes) del pedido. \textit{Entero}.
          %       \item \texttt{weight}: Peso del pedido. \textit{Decimal}.
          %       \item \texttt{notes}: Notas del pedido. \textit{Cadena de caracteres}.
          %       \item \texttt{origin}: Origen del pedido (creado automáticamente, importado, manual). \textit{Entero}.
          %       \item \texttt{updated\_at}: Fecha de la última actualización del pedido. \textit{Fecha y hora}.
          %       \item \texttt{carrier\_id}: Identificador del transportista del pedido. \textit{Entero y relación N:1 con la tabla} \texttt{carrier}.
          %       \item \texttt{customer\_id}: Identificador del cliente del pedido. \textit{Entero y relación N:1 con la tabla} \texttt{customer}.
          %       \item \texttt{marketplace\_id}: Identificador del canal de venta del pedido. \textit{Entero y relación N:1 con la tabla} \texttt{marketplace}.
          %   \end{itemize}
          \begin{table}[H]
    \centering
    \begin{tabular}{l p{4cm} p{3cm} l}
        \textbf{Campo}             & \textbf{Descripción}                          & \textbf{Tipo de dato} & \textbf{Relación}            \\ \hline \hline
        \texttt{id}                & Identificador único del pedido                & Entero                & Clave primaria               \\
        \texttt{order\_id}         & Identificador del pedido en el canal de venta & Cadena caracteres     &                              \\
        \texttt{status}            & Estado del pedido                             & Entero                &                              \\
        \texttt{order\_date}       & Fecha en la que se realizó el pedido          & Fecha y hora          &                              \\
        \texttt{total\_price}      & Importe total del pedido                      & Decimal               &                              \\
        \texttt{ticket}            & Número de ticket del pedido                   & Cadena caracteres     &                              \\
        \texttt{ticket\_refund}    & Número de ticket de la devolución del pedido  & Cadena caracteres     &                              \\
        \texttt{pay\_method}       & Método de pago del pedido                     & Entero                &                              \\
        \texttt{package\_quantity} & Cantidad de paquetes del pedido               & Entero                &                              \\
        \texttt{weight}            & Peso del pedido                               & Decimal               &                              \\
        \texttt{notes}             & Notas del pedido                              & Texto                 &                              \\
        \texttt{origin}            & Origen del pedido                             & Entero                &                              \\
        \texttt{updated\_at}       & Fecha de la última actualización del pedido   & Fecha y hora          &                              \\
        \texttt{carrier\_id}       & Identificador del transportista del pedido    & Entero                & N:1 con \texttt{carrier}     \\
        \texttt{customer\_id}      & Identificador del cliente del pedido          & Entero                & N:1 con \texttt{customer}    \\
        \texttt{marketplace\_id}   & Identificador del canal de venta del pedido   & Entero                & N:1 con \texttt{marketplace}
    \end{tabular}
\end{table}

    \item \textbf{Cliente [\texttt{customer}]:} Esta tabla almacena la información del cliente. Los campos que contiene son los siguientes:
\end{itemize}

\begin{figure}
    \centering
    \includegraphics[width=0.98\textwidth]{figures/design_develop/database_diagram.pdf}
    \caption{Diagrama de la base de datos}
    \label{fig:diagrama_base_datos}
\end{figure}

\section{Desarrollo del \textit{backend}}


\section{Desarrollo del \textit{frontend}}
\chapter{Conclusiones}
\label{chap:conclusiones}

La realización de este proyecto ha llevado al desarrollo de una plataforma web para la gestión centralizada de múltiples canales de venta en línea o \textit{marketplaces}. Consecuentemente, se ha permitido demostrar la viabilidad técnica y funcional de una herramienta que tiene capacidad de automatizar y simplificar una parte muy importante de las tareas de gestión de un negocio que opera en múltiples plataformas de venta. La solución desarrollada aborda problemáticas comunes tales como la fragmentación de la información, la dificultad de mantener y supervisar múltiples canales de venta y la necesidad de optimizar recursos para pequeños y medianos comercios que desean expandir su presencia en línea sin incurrir en complicaciones técnicas excesivas.

Desde su concepción, la plataforma ha sido diseñada con un enfoque claro en las necesidades de los pequeños y medianos negocios, que a menudo carecen de los recursos técnicos o humanos para gestionar eficientemente la presencia de su producto en línea. La simplicidad de uso, la automatización de procesos repetitivos y la accesibilidad de la aplicación buscan empoderar a este tipo de negocios, permitiéndoles competir en prácticamente igualdad de condiciones dentro del entorno del comercio electrónico global. De este modo, se contribuye a la digitalización del tejido comercial minorista, reduciendo barreras de entrada y fomentando su crecimiento sostenible.

Si bien es cierto que el proyecto se encuentra lejos de ser una solución completa, se ha tratado de aplicar buenas prácticas para garantizar su escalabilidad y su capacidad de adaptación a futuras necesidades. Se han implementado tecnologías modernas para asegurar una continuidad en el desarrollo del proyecto, permitiendo que se pueda seguir evolucionando y mejorando con el tiempo.

Ha sido un reto significativo, no solo por el desarrollo técnico de la aplicación, sino también por la gestión de un proyecto que en un futuro podría ser de gran envergadura. Es complicado detallar todo el proceso de desarrollo y las decisiones tomadas, pues ha sido un proceso iterativo y evolutivo, donde cada decisión ha influido en las siguientes. Sin embargo, se ha procurado documentar adecuadamente cada fase del proyecto para que pueda ser entendido y replicado en el futuro, hasta para aquellos que no tengan experiencia previa en el desarrollo de aplicaciones web.

En definitiva, este trabajo constituye un primer paso hacia una solución integral para la gestión multicanal de comercio electrónico, con la posibilidad de evolucionar y consolidarse como un posible negocio en el futuro. Los resultados obtenidos confirman que la idea inicial es viable y que la plataforma puede ser una herramienta valiosa para los pequeños y medianos comercios que buscan una alternativa para reducir la complejidad operativa resultado de su presencia en línea. Esta primera versión desarrollada sienta una base sólida para futuras mejoras, ampliaciones de funcionalidades y adaptaciones a las necesidades cambiantes del mercado.

\section{Trabajo futuro}

Aunque la plataforma desarrollada constituye una primera versión funcional, lejos se encuentra de ser una solución completa. Por ello, se han identificado diversas áreas de mejora y ampliación que podrían ser objeto de futuros trabajos. Algunas de las más relevantes son:

\begin{itemize}
    \item Implementación de un sistema de autenticación que permita a los usuarios tener su propia cuenta para así gestionar sus canales de venta.
    \item Implementación de un sistema de gestión de envíos más complejo, que permita a los usuarios crear envíos, tratar incidencias, crear etiquetas, entre muchos otros.
    \item Ampliación de la funcionalidad de análisis de ventas, permitiendo a los usuarios obtener informes más detallados y personalizados sobre sus ventas y productos.
    \item Integración con más plataformas de venta, ampliando así el número de canales que los usuarios pueden gestionar desde la aplicación.
    \item Implementación de una \gls{api} que permita a terceros desarrollar aplicaciones que se integren con la plataforma, pudiendo así vincular la aplicación con sistemas externos como \gls{crm} o sistemas de gestión de inventario.
    \item Implementar una funcionalidad de importación y exportación de datos, permitiendo a los usuarios subir productos desde archivos CSV o Excel, así como exportar sus datos a formatos comunes para su análisis externo.
\end{itemize}

Con todas estas implementaciones, se podría obtener una plataforma mucho más completa y funcional, lista para poder ser desplegada en un servidor y empezar a ser comercializada. A pesar de ser un proceso complejo y laborioso, la base sentada durante este proyecto es sólida y ha demostrado que la idea inicial es viable. Por lo tanto, el futuro del proyecto es prometedor y se espera que pueda seguir evolucionando y mejorando con el tiempo.

\newpage
\printbibliography

\end{document}